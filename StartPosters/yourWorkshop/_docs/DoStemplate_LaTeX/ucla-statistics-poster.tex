\documentclass[landscape,a0paper,fontscale=0.34]{baposter}
\usepackage{amssymb}
\usepackage{amsmath}
\usepackage{amsthm}
%\usepackage{amsaddr} % to allow inclusion of institution address
%\usepackage{charter}
\usepackage{enumitem}
%\usepackage{wrapfig}
\usepackage{caption}
\usepackage{floatrow}  % to place graph and caption side by side
\usepackage{multicol}
\usepackage{listings}


%\usepackage{tikz}
%\usepackage{tkz-graph}

%\usepackage{subcaption} % for better subcaptions
%\usepackage{verbatim}   % for comment environment
%\usepackage{booktabs}   % for publication-quality tables
%\usepackage{algorithm}  % for the algorithm environment
%\usepackage[flushleft, online]{threeparttable} % for table footnotes
%\usepackage{grffile}

%\usepackage[usenames,dvipsnames]{color} % for highlighting / strikethroughs
%\usepackage{multicol}
%\definecolor{boxcolour}{RGB}{120,245,230}
\definecolor{boxcolour}{RGB}{191, 250, 238}

%%%%%%%%%%%%%%%%%%%%%%%%%%%%%%%%%%%%%%%%%%%%%%%%%%%%%%%%%%%%%%%%%%%%%%%%%%%%%%%%
% Multicol Settings
%%%%%%%%%%%%%%%%%%%%%%%%%%%%%%%%%%%%%%%%%%%%%%%%%%%%%%%%%%%%%%%%%%%%%%%%%%%%%%%%
\setlength{\columnsep}{0.7em}
\setlength{\columnseprule}{0mm}

%%%%%%%%%%%%%%%%%%%%%%%%%%%%%%%%%%%%%%%%%%%%%%%%%%%%%%%%%%%%%%%%%%%%%%%%%%%%%%%%
% Save space in lists. Use this after the opening of the list
%%%%%%%%%%%%%%%%%%%%%%%%%%%%%%%%%%%%%%%%%%%%%%%%%%%%%%%%%%%%%%%%%%%%%%%%%%%%%%%%
\newcommand{\xcompresslist}{%
 \setlength{\itemsep}{1pt}%
 \setlength{\parskip}{0pt}%
 \setlength{\parsep}{0pt}%
}

\newcommand{\compresslist}{%
 \setlength{\itemsep}{5pt}%
 \setlength{\parskip}{0pt}%
 \setlength{\parsep}{0pt}%
}

\newcommand{\compresslistless}{%
 \setlength{\itemsep}{3.25pt}%
 \setlength{\parskip}{0pt}%
 \setlength{\parsep}{0pt}%
}

%Special for paper
\def\phi{\varphi}
\def\by{\mathbf{y}}
\def\bs{\mathbf{\sigma}}
\def\bX{\mathbf{X}}
\def\bB{\mathbf{B}}
\def\bPhi{\mathbf{\Phi}}
\def\brho{\mathbf{\rho}}
\def\pl{p_\lambda}
\def\pln{p_{\lambda_n}}
\def\G{\mathcal{G}}
\def\rto{\leftarrow}
\def\s{\sigma}
\def\o{\omega}
\def\A{\mathcal{A}}
\renewcommand{\th}{\theta}
%\renewcommand{\thh}{\hat{\theta}}
\newcommand{\bth}{\boldsymbol\theta}
\renewcommand{\b}{\beta}
\newcommand{\hb}{\hat{\beta}}
\newcommand{\hbB}{\widehat{\mathbf{B}}}
\newcommand{\hB}{\widehat{B}}
\newcommand{\hO}{\widehat{\Omega}}
\newcommand{\hP}{\widehat{\Phi}}
\newcommand{\hR}{\widehat{R}}
\newcommand{\hT}{\widehat{\Theta}}
\newcommand{\one}{\mathbf{1}}
\def\bn{\boldsymbol\nu}
\def\bu{\mathbf{u}}
\newcommand{\tB}[1]{\widetilde{B}(#1)}
\newcommand{\tOm}[1]{\widetilde{\Omega}(#1)}

%%%%%%%%%%%%%%
\def\estperm{\widehat{\pi}}
\def\dagadjest{\widehat{B}}
\def\goodperm{\pi_{0}}
\def\penderiv{\rho_\lambda'(0+)}


\definecolor{test}{rgb}{0,50,0}
\newcommand{\cemph}[1]{\textcolor{cyan!80!blue}{#1}}
\newcommand{\ccemph}[1]{\textcolor{magenta}{#1}}

%%%%%%%%%%%%%%%%%%%%%%%%%%%%%%%%%%%%%%%%%%%%%%%%%%%%%%%%%%%%%%%%%%%%%%%%%%%%%
%% Begin of Document
%%%%%%%%%%%%%%%%%%%%%%%%%%%%%%%%%%%%%%%%%%%%%%%%%%%%%%%%%%%%%%%%%%%%%%%%%%%%%
\begin{document}
%%%%%%%%%%%%%%%%%%%%%%%%%%%%%%%%%%%%%%%%%%%%%%%%%%%%%%%%%%%%%%%%%%%%%%%%%%%%%
%% Here starts the poster
%%---------------------------------------------------------------------------
%% Format it to your taste with the options
%%%%%%%%%%%%%%%%%%%%%%%%%%%%%%%%%%%%%%%%%%%%%%%%%%%%%%%%%%%%%%%%%%%%%%%%%%%%%
\begin{poster}{
 % Show grid to help with alignment
 grid=false,
 columns=3,
 % Column spacing
 colspacing=1em,
 % Color style
 headerColorOne=boxcolour, %gray!20!white!90!black,
 borderColor=boxcolour,
 % Format of textbox
 textborder=rectangle,
 linewidth=0.25mm,
 % Format of text header
 headerborder=closed,
 headershape=rectangle,
 headershade=plain,
 headerfont=\large\rmfamily\bf,
 background=none,
 bgColorOne=gray!90!black,
 boxColorOne=white,
 headerheight=0.12\textheight}
 % Eye Catcher
 {
 % empty
     
 }
 % Title
 {\huge Poster Title}
 % Authors
{Ronald A. Fisher \\[0.2em]{(*\texttt{xxx@stat.ucla.edu})}}
 % University logo
 {
  \begin{tabular}{c}
    \includegraphics[height=0.09\textheight]{ps_statistics_DoS}%{ucla_cw}
  \end{tabular}
 }

%%%%%%%%%%%%%%%%%%%%%%%%%%%%%%%%%%%%%%%%%%%%%%%%%%%%%%%%%%%%%%%%%%%%%%%%%%%%%%
%%% Now define the boxes that make up the poster
%%%---------------------------------------------------------------------------
%%% Each box has a name and can be placed absolutely or relatively.
%%% The only inconvenience is that you can only specify a relative position 
%%% towards an already declared box. So if you have a box attached to the 
%%% bottom, one to the top and a third one which should be inbetween, you 
%%% have to specify the top and bottom boxes before you specify the middle 
%%% box.
%%%%%%%%%%%%%%%%%%%%%%%%%%%%%%%%%%%%%%%%%%%%%%%%%%%%%%%%%%%%%%%%%%%%%%%%%%%%%%

%%%%%%%%%%%%%%%%%%%%%%%%%%%%%%%%%%%%%%%%%%%%%%%%%%%%%%%%%%%%%%%%%%%%%%%%%%%%%%
\headerbox{Objectives \& Motivations}{name=problem,column=0,row=0,span=1}{
	\compresslistless
	\begin{itemize}
		\item Develop novel and modern techniques in high-dimensional statistics and sparse regularization for structure learning of Bayesian networks (BNs) from big data.
		\item Inspired by applications in computational biology, e.g., construction of gene networks from high-throughput genomic data. Hence, focused on methods that scale to thousands of variables for both continuous and discrete data.
	\end{itemize}
}

%%%%%%%%%%%%%%%%%%%%%%%%%%%%%%%%%%%%%%%%%%%%%%%%%%%%%%%%%%%%%%%%%%%%%%%%%%%%%%
% \headerbox{High-Dimensonal Theory}{name=theory, column=0, span=1, below=problem}{
% Theory part of high-dimensional Bayesian networks (BNs).
% }

%%%%%%%%%%%%%%%%%%%%%%%%%%%%%%%%%%%%%%%%%%%%%%%%%%%%%%%%%%%%%%%%%%%%%%%%%%%%%%
\headerbox{Penalized Likelihood and High-D Theory}{name=framework, column=0, row=0, span=1, below=problem}{
%\begin{wrapfigure}{r}{4cm}

%\end{wrapfigure}
\begin{minipage}{.75\textwidth}
	\begin{itemize}
		\xcompresslist
		\item $n$ i.i.d. observations of $X=(X_1,\ldots,X_p)$; \\ adding interventions for causal learning is possible.
		\item Bayesian network (BN) for $X$, parameterized by coefficients $B:=(\beta_{ij})$, interpreted as a weighted adjacency matrix, \emph{always a directed acyclic graph (DAG)}.
		\item Item ...
	
	\end{itemize}
\end{minipage}
\begin{minipage}{.2\textwidth}
	%\includegraphics[scale=.85]{figs/dag_example2.pdf}
\end{minipage}



\textbf{Theory:} High-D regime, $p\gg n\to\infty$, degree-growth $d \log p/ n = o(1)$,
\begin{enumerate}%[leftmargin=0cm, itemindent=.5cm, itemsep=0em]
	\compresslistless
	\item \textbf{Deviation bounds:} 
	
	\item \textbf{Sparsity bounds:} 
	\item \textbf{Model selection consistency:} 
	
	\item \textbf{Uniform control of SEM coeffs.:} All estimated DAGs $\widehat B(\pi)$ are close to their true DAG $\widetilde{B}(\pi)$, for all orderings $\pi$. Leveraged \cemph{lattice property} of neighborhood regression and led to study of abstract neighborhood regression and connections with PCGs and their algebraic  properties. 
\end{enumerate}

}

\headerbox{PCGs and neighborhood regression}{name=pcg,column=1, row=0, span=1}{
	\begin{itemize}
		\xcompresslist
		\item General framework for studying \cemph{neighborhood regression}. Regressing $X_j$ on a subset $S$ of the rest of variables $X_S = \{X_i, i \in S\}$:
		
		\item \cemph{Partial correlation graphs (PCGs)}: Natural setting for studying these relations. General setup in a Hilbert space. Express everything in terms of
		\end{itemize}
		
}

%%%%%%%%%%%%%%%%%%%%%%%%%%%%%%%%%%%%%%%%%%%%%%%%%%%%%%%%%%%%%%%%%%%%%%%%%%%%%%
%\headerbox{References}{name=software,column=2,span=1, below = discrete}{
\headerbox{References}{name=software,column=0,span=1, below = framework}{
\small
Paper 1 \\
 Paper 2\\
Preprint 1, arXiv:xxxx.

}

\end{poster}%
\end{document}
